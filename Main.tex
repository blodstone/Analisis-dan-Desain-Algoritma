\documentclass[a4paper]{tufte-book}
\usepackage{lmodern}% http://ctan.org/pkg/lm
\newcounter{ccounter}
\newcounter{clatihan}
\newcounter{clisting}
\renewcommand\theccounter{\arabic{ccounter}}
\renewcommand\theclatihan{\arabic{clatihan}}
\renewcommand\theclisting{\arabic{clisting}}
\usepackage{soul,color}
\usepackage{placeins}
\usepackage{algorithmic}
\usepackage{algorithm}
\usepackage{listings}
\usepackage{lstlinebgrd}
\usepackage{graphicx}
\usepackage{makeidx}
\usepackage{mathtools}
\usepackage{bookmark}
\usepackage{xcolor}
\usepackage{mathpazo}
\usepackage{multirow}
\definecolor{codegreen}{rgb}{0,0.6,0}
\definecolor{codegray}{rgb}{0.5,0.5,0.5}
\definecolor{codepurple}{rgb}{0.58,0,0.82}
\definecolor{backcolour}{rgb}{0.95,0.95,0.92}
\definecolor{codehighlight}{rgb}{1,0.65,0.65}
 
\lstdefinestyle{pylist}{
    backgroundcolor=\color{backcolour},   
    commentstyle=\color{codegreen},
    keywordstyle=\color{magenta},
    numberstyle=\tiny\color{codegray},
    stringstyle=\color{codepurple},
    basicstyle=\footnotesize,
    breakatwhitespace=false,         
    breaklines=true,                 
    captionpos=b,                    
    keepspaces=true,                 
    numbers=left,                    
    numbersep=5pt,                  
    showspaces=false,                
    showstringspaces=false,
    showtabs=false,                  
    tabsize=2
}
 
\lstset{style=pylist}

\author{STMIK Mikroskil}
\title{Desain dan Analisis Algoritma}
\publisher{Teknik Informatika -- STMIK Mikroskil}
\newenvironment{myindentpar}[1]%
{
	\begin{list}{}%
  {
  	\setlength{\leftmargin}{#1}}%
    \item[]%
	}
{\end{list}}

\newenvironment{contoh}{
	\refstepcounter{ccounter}
	\begin{myindentpar}{1cm}
	\textbf{\underline{Contoh \arabic{ccounter}}}
}{	
	\end{myindentpar}
}

\newenvironment{latihan}{
	\refstepcounter{clatihan}
	\begin{myindentpar}{1cm}
	\textbf{\underline{Latihan \arabic{clatihan}}}	
}{	
	\end{myindentpar}
}

\newenvironment{konsep}{
	\begin{latihan}
	\small\textbf{(Konsep)}
}{
	\end{latihan}
}

\newenvironment{teori}{
    \begin{myindentpar}{1cm}
    \textbf{\underline{Teori}}
}{
    \end{myindentpar}
}

\newenvironment{pemrograman}{
	\begin{latihan}
	\small\textbf{(Pemrograman)}
}{
	\end{latihan}
}

\newenvironment{proyek}{
	\begin{latihan}
	\small\textbf{(Kelompok)}
}{
	\end{latihan}
}

\newenvironment{listprog}[1]{
	\refstepcounter{clisting}
	\begin{myindentpar}{1cm}
	\textbf{\underline{Listing \arabic{clisting}} {#1}}
	
}{
	\end{myindentpar}
}

\renewcommand{\lstlistingname}{Algoritma}% Listing -> Algoritma
\renewcommand{\lstlistlistingname}{Daftar \lstlistingname s}% List of Listings -> Daftar Algoritma

\makeatletter

\makeatletter

\makeindex
\begin{document}

\maketitle
%\tableofcontents
%\renewcommand{\chaptername}{Modul}
\renewcommand{\figurename}{Gambar}
\floatname{algorithm}{Algoritma}

%\chapter{Graph}

Graph merupakan sebuah \textit{Abstract Data Type} (ADT) yang digunakan untuk mengimplementasikan konsep \textit{graph} dan \textit{directed graph} dalam matematika. Dalam matematika sendiri, graph merupakan sebuah representasi dari sekumpulan objek yang saling terhubung. Objek-objek yang ada di dalam graph dikenal dengan nama \textit{vertex}, sedangkan hubungan (penghubung) antar objek tersebut dikenal dengan nama \textit{edge}.

Sebuah graph dapat digunakan untuk merepresentasikan banyak hal dalam dunia nyata. Misalnya, kita dapat merepresentasikan pohon pengetahuan dalam sebuah graph seperti yang nampak pada gambar~\ref{fig:knowledge-tree}. Peta juga kerap kali direpresentasikan sebagai graph, dengan titik-titik pergerakan sebagai vertex-nya, dan jalur antar titik sebagai edge-nya (lihat gambar~\ref{fig:map-graph}).

\begin{figure}
    \includegraphics[width=\textwidth,keepaspectratio]{fig/KnowledgeTree.png}%
	\caption{Pohon Pengetahuan}%
	\label{fig:knowledge-tree}%
\end{figure}

\begin{figure}
    \includegraphics[width=\textwidth,keepaspectratio]{fig/DirectedGraphMap.png}%
	\caption{Peta dengan Graph}%
	\label{fig:map-graph}%
\end{figure}

Terdapat sangat banyak jenis dan definisi dari graph, yang masing-masing memiliki kegunaan spesifik dan kelebihan serta kekurangan tersendiri. Pada bagian ini kita hanya akan membahas satu jenis graph, yaitu graph tidak berarah yang berlabel. Untuk mengetahui lebih lanjut mengenai detil dari berbagai jenis graph serta kelebihan dan kekurangannya, silahkan baca buku atau modul tentang struktur data terkait.

Graph tidak berarah berlabel yang kita gunakan didefinisikan sebagai berikut:

\begin{itemize}
    \item Sebuah graph didefinisikan sebagai $G = (V, E)$.
    \item $V$ merupakan sekumpulan vertex.
    \item $E$ merupakan sekumpulan edge.
    \item $E = (V1, V2, v)$ di mana $V1$ dan $V2$ adalah dua buah vertex yang terhubung dan $v$ adalah label (bobot; jarak) dari kedua vertex tersebut.
    \item Dua buah vertex yang saling berdampingan membentuk sebuah edge dapat dihubungkand engan simbol $~$, sehingga $u ~ v$ dapat dibaca sebagai vertex $u$ dan $v$ yang berdampingan (memiliki edge).
\end{itemize}

Definisi graph yang kita gunakan tidak terlalu jauh berbeda dengan yang digunakan pada teori graph dalam matematika pada umumnya. Tetapi ingat bahwa definisi ini seringkali dimodifikasi sesuai dengan kebutuhan dan tujuan dari algoritma yang menggunakan graph tersebut. Misalnya, representasi dan definisi dari sebuah graph yang digunakan untuk menyelesaikan permasalahan pemetaan seperti mencari jalur terpendek akan berbeda dengan representasi untuk menyelesaikan masalah deteksi bahasa. Begitupun, algoritma-algoritma dasar yang sama dapat kita immplementasikan pada representasi graph yang berbeda ini (misalnya: algoritma untuk pencarian jalur terpendek). Perbedaan hanya akan ditemukan pada detil implementasi nantinya.

Untuk memperjelas pengertian tentang representasi graph yang berbeda, kita akan melihat beberapa jenis cara merepresentasikan graph yang umum digunakan.

\section{Representasi Graph}

Secara umum terdapat tiga metode untuk merepresentasikan graph dalam ilmu komputer, yaitu:

\begin{enumerate}
    \item Adjacency List
    \item Adjacency Matrix
    \item Incidence Matrix
\end{enumerate}

\subsection{Graph dengan Adjacency List}

\subsection{Graph dengan Adjacency Matrix}

\subsection{Graph dengan Incidence Matrix}

\section{Operasi Umum Graph}

Terdapat beberapa operasi umum yang dapat dilakukan terhadap graph, yaitu:

\begin{enumerate}
    \item Penambahan Vertex baru
    \item Penambahan Edge baru
    \item Penghapusan Vertex
    \item Penghapusan Edge
    \item Pengecekan apakah dua buah vertex terhubung
\end{enumerate}

Kompleksitas dari masing-masing operasi sendiri berbeda-beda, tergantung dari cara representasi graph yang kita gunakan

\section{Contoh Implementasi Graph}

%\chapter{Graph Traversal}

Permasalahan paling mendasar dari sebuah graph adalah mengunjungi setiap edge dan vertex dari sebuah graph dalam cara yang sistematik. Untuk menyelesaikan permasalahan tersebut maka kita memerlukan sebuah algoritma graph traversal. Ide dasar dari sebuah graph traversal adalah menandai (marking) setiap vertex yang telah kita kunjungi dan menjelajahi vertex yang belum kita kunjungi.

Pada umumnya, kita beri 3 tanda pada setiap vertex sebagai berikut.
\begin{enumerate}
	\item Belum dikunjungi (\textit{undiscovered}), vertex yang masih dalam kondisi awal. Umumnya ditandai dengan warna putih.
	\item Sudah dikunjungi (\textit{discovered}), vertex yang sudah dikunjungi tetapi masih belum menjelajahi tetangga dari vertex tersebut. Umumnya ditandai dengan warna abu-abu.
	\item Sudah diproses (\textit{processed}), vertex yang sudah dikunjungi dan semua tetangga juga sudah dijelajahi. Umumnya ditandai dengan warna hitam.
\end{enumerate}

Sebuah vertex berubah tanda secara bertahap dari \textit{Undiscovered} menjadi \textit{Discovered} dan terakhir menjadi \textit{Processed}. 

Ada 2 algoritma Graph Traversal yang populer yaitu \textit{Breadth-First Search} dan \textit{Depth-First Search}. Masih ada banyak metode Graph Traversal yang lain tetapi itu tidak dibahas di materi ini. Mahasiswa diharapkan untuk mencari tau jenis-jenis Graph Traversal yang lain.

\section{Breadth-First Search}

\textit{Breadth-First Search} (BFS) merupakan metode Graph Traversal yang paling sederhana dan menjadi dasar dari banyak algoritma graph lainnya seperti algoritma Djikstra untuk mencari jalur terpendek dan algoritma Prim untuk mencari minimum spanning tree.

Diberikan sebuah graph $G = (V,E)$ dan sebuah vertex awal $s$, maka algoritma \textit{Breadth-First Search} akan menjelajahi setiap edge dari G untuk menemukan semua vertex yang bisa dicapai dari $s$. \textit{Breadth-First Search} juga dapat mencari jumlah edge minimal (tanpa menghitung bobot edge) untuk mencapai sebuah vertex tertentu. Algoritma BFS bisa digunakan di graph berarah atau tidak berarah. Kompleksitas waktu dari BFS adalah O(V+E).

Illustrasi BFS bisa dilihat di ~\ref{fig:BFS}.

\begin{figure}
    \includegraphics[width=\textwidth,keepaspectratio]{fig/BFS.png}%
	\caption{Illustrasi BFS}%
	\label{fig:BFS}%
\end{figure}

\subsection{Implementasi Breadth-First Search}

Implementasi lengkap dari BFS bisa dilihat dari Algo ~ref{algo:migraph-BFS}.

\lstinputlisting[language=Python, 
                 firstline=117,
                 lastline=148,
                 label={algo:migraph-BFS},
                 caption=Implementasi BFS
                ]
                {code/migraph.py}

Fungsi BFS akan menerima satu parameter yaitu vertex sumber $s$ dimana kita akan mulai menjelajahi.

\lstinputlisting[language=Python, 
                 firstline=117,
                 lastline=117,
                 label={algo:migraph-BFS-def},
                 caption=Definisi fungsi BFS
                ]
                {code/migraph.py}

Sebelum memasuki inti dari algoritma BFS, pertama kita deklarasi beberapa variabel untuk BFS yang bisa dilihat di Algo ~\ref{algo:migraph-variabel-BFS}. Variabel $i$, digunakan ketika kita akan mencetak langkah traversal dari BFS. Dengan variabel $i$ kita bisa mengetahui vertex mana yang dikunjungi terlebih dahulu dan selanjutnya. Variabel $traversal$ untuk menyimpan semua urutan kunjungan vertex. Kedua variabel $i$ dan $traversal$ hanya untuk keperluan cetak dan bukan bagian dari BFS. 

Variabel $\_\_color$ digunakan untuk mewarnai setiap vertex yang dikunjungi. Ada tiga warna yaitu $WHITE$, $GRAY$ dan $BLACK$. Variabel $\_\_distance$ untuk mengetahui jarak dari sebuah vertex dari source (vertex awal). Variabel $\_\_predecessor$ untuk mengetahui \textit{predecessor} (parent) dari sebuah vertex. 

\lstinputlisting[language=Python, 
                 firstline=118,
                 lastline=122,
                 label={algo:migraph-variabel-BFS},
                 caption=Variabel BFS
                ]
                {code/migraph.py}

Setelah kita deklarasi, kita definisikan variabel $\_\_color$ dan $\_\_distance$. Semua vertex awalnya berwarna putih dan jaraknya INF. $\_\_predecessor$ tidak usah didefinisikan karena memang nilai awalnya adalah NONE.

\lstinputlisting[language=Python, 
                 firstline=123,
                 lastline=125,
                 label={algo:migraph-define-variabel-BFS},
                 caption=Definisi Variabel BFS
                ]
                {code/migraph.py}

Kita definisikan warna dari vertex $s$ sebagai warna $GRAY$ menandakan vertex sudah dikunjungi tetapi belum semua tetangga dikunjungi dan jarak bernilai 0.

\lstinputlisting[language=Python, 
                 firstline=126,
                 lastline=127,
                 label={algo:migraph-define-s-BFS},
                 caption=Definisi Variabel S BFS
                ]
                {code/migraph.py}

Untuk melacak vertex mana yang sudah dikunjungi kita akan gunakan \textit{Queue} $Q$. Di python, untuk menyederhanakan kita bisa menggunakan \textit{list} yang bisa dilihat di Algo ~\ref{algo:migraph-queue-BFS}. Kita masukkan dulu vertex $s$ ke dalam \textit{Queue} menandakan kita sedang mengunjungi vertex $s$.

\lstinputlisting[language=Python, 
                 firstline=128,
                 lastline=129,
                 label={algo:migraph-queue-BFS},
                 caption=Penggunaan Queue BFS
                ]
                {code/migraph.py}

$Q$ berfungsi sebagai penanda kunjungan kita ke setiap vertex. Semua vertex yang akan kita kunjungi akan kita masukkan dari belakang (\textit{append}) variabel $Q$. Semua tetangga dari vertex $s$ akan kita \textit{append} ke dalam $Q$. Setelah itu kita akan proses dari depan satu persatu. Setiap kita proses satu vertex, kita \textit{append} semua tetangga dari vertex tersebut untuk dikunjungi nanti.

Kita akan looping terus menerus dan lakukan proses di atas sampai semua vertex sudah dikunjungi atau $Q$ tidak berisi lagi (lihat ~\ref{algo:migraph-queue-while-BFS}).

\lstinputlisting[language=Python, 
                 firstline=131,
                 lastline=131,
                 label={algo:migraph-queue-while-BFS},
                 caption=Looping selama masih ada isi di Queue.
                ]
                {code/migraph.py}
								
Di awal dari looping, kita akan mengeluarkan vertex paling depan dari $Q$ untuk diproses. 

\lstinputlisting[language=Python, 
                 firstline=132,
                 lastline=132,
                 label={algo:migraph-queue-pop-BFS},
                 caption=Keluarkan isi vertex yang berada di tempat terdepan Queue.
                ]
                {code/migraph.py}

Kita akan memproses vertex yang dikeluarkan ($u$) jika warna dari vertex $u$ adalah $WHITE$ (artinya belum pernah dikunjungi sebelumnya). Pertama kita akan melihat semua tetangga dari vertex $u$. Setiap tetangga tersebut ($v$) kita akan beri warna $GRAY$ menandakan sudah dikunjungi tetapi belum mengunjungi tetangganya. Kemudian kita tambah jarak vertex tersebut 1 (jarak menandakan jarak dari $s$ ke vertex tersebut). Kemudian kita set \textit{predecessor} dari semua vertex tetangga menjadi $u$. Yang terakhir semua tetangga dimasukkan ke belakang dari $Q$.

\lstinputlisting[language=Python, 
                 firstline=138,
                 lastline=146,
                 label={algo:migraph-queue-process-BFS},
                 caption=Proses vertex yang dikeluarkan dari Queue.
                ]
                {code/migraph.py}

Setelah diproses, vertex $u$ tersebut diberi warna $BLACK$ yang artinya sudah dikunjungi dan semua tetangga juga sudah dikunjungi.

\lstinputlisting[language=Python, 
                 firstline=147,
                 lastline=147,
                 label={algo:migraph-vertex-Black-BFS},
                 caption=Vertex diberi warna hitam.
                ]
                {code/migraph.py}
								
Setiap kali kita memproses sebuah vertex, kita akan simpan ke dalam variabel $traversal$ untuk mencatat kunjungan kita. Baris ini bukan bagian dari BFS tetapi hanya untuk keperluan mencatat saja.

\lstinputlisting[language=Python, 
                 firstline=133,
                 lastline=137,
                 label={algo:migraph-traversal-BFS},
                 caption=Mencatat kunjungan dalam traversal.
                ]
                {code/migraph.py}
								
\section{Depth-First Search}

Strategi dari \textit{Depth-First Search} (DFS) berbeda dari BFS dimana BFS mengunjungi terlebih dahulu semua tetangga, maka DFS mengunjungi satu vertex secara mendalam sampai tidak bisa dikunjungi lagi baru pindah ke tetangga baru. Kompleksitas waktu dari DFS sama seperti BFS, O(V+E), akan tetapi untuk graph yang sangat besar sekali DFS bisa terlalu dalam menjelajahi satu verteks tanpa menjelajahi verteks lain.

Illustrasi DFS bisa dilihat di ~\ref{fig:Dfs}.

\begin{figure}
    \includegraphics[width=\textwidth,keepaspectratio]{fig/DFS.png}%
	\caption{Illustrasi DFS}%
	\label{fig:Dfs}%
\end{figure}


\subsection{Implementasi Depth-First Search}

Implementasi lengkap dari BFS bisa dilihat dari Algo ~ref{algo:migraph-DFS}. Algoritma dari DFS merupakan algoritma rekursif.

\lstinputlisting[language=Python, 
                 firstline=88,
                 lastline=115,
                 label={algo:migraph-DFS},
                 caption=Implementasi DFS
                ]
                {code/migraph.py}

Pada fungsi $DFS\_traversal(s)$, semua verteks akan diwarnai dengan warna $WHITE$ terlebih dahulu. Setelah itu kita akan mulai mengunjungi dari vertex $s$. Kunjungan akan ditandai dengan memanggil fungsi $DFS\_visit$

\lstinputlisting[language=Python, 
                 firstline=88,
                 lastline=98,
                 label={algo:migraph-DFS-2},
                 caption=Implementasi DFS
                ]
                {code/migraph.py}

Fungsi $DFS\_visit$ akan mengunjungi semua vertex beserta anaknya sampai yang paling dalam. Inti dari DFS adalah pada rekursif, dimana setiap tetangga vertex $v$ yang berwarna $WHITE$ akan dikunjungi dan ketika rekursif selesai, vertex $v$ tersebut akan diwarnai $BLACK$.

\lstinputlisting[language=Python, 
                 firstline=100,
                 lastline=115,
                 label={algo:migraph-DFS-3},
                 caption=Implementasi DFS
                ]
                {code/migraph.py}


\chapter{Pemrosesan String}

\section{Pengenalan}

String merupakan sebuah deretan simbol yang bisa dituliskan dengan menggunakan notasi $a_1 a_2 \ldots a_k$, atau $(a_1, a_2, \ldots , a_k)$. Pemrosesan String sangat berguna dalam berbagai aplikasi seperti pemrosesan genom, sistem komunikasi, pemrosesan informasi, dan sebagainya. Ada beberapa struktur data dan algoritma yang harus dipahami terlebih dahulu sebelum melakukan pemrosesan string.  

\section{Struktur Data yang Berkaitan dengan String}
Struktur data yang akan dibahas adalah \textit{Trie}, \textit{Suffix trie}, \textit{Suffix tree} dan \textit{Suffix array}.

\subsection{\textit{Trie}}
\textit{Trie} merupakan singkatan dari kata \textit{reTRIEval} dan merupakan sebuah varian dari pohon pencarian. \textit{Trie} terdiri dari satu atau lebih \textit{node} (\textit{vertex}). Setiap \textit{node} memiliki hubungan (\textit{link}) ke \textit{node} lain ataupun null. Setiap \textit{node} memiliki satu \textit{parent node} dan setiap \textit{node} memiliki satu atau beberapa \textit{link} (\textit{edge}). Contoh dari Trie bisa dilihat di Gambar ~\ref{fig:trie}.

\begin{figure}
    \includegraphics[width=\textwidth,keepaspectratio]{fig/Trie.png}%
	\caption{Pohon Pengetahuan}%
	\label{fig:trie}%
\end{figure}

Gambar \ref{fig:trie} dibentuk dari 4 buah string yaitu: $makan$ bernilai 10, $masak$ bernilai 14, $minum$ bernilai 20 dan $tidur$ bernilai 11. Untuk mencari string $makan$, maka pencarian dimulai dari huruf awal yaitu $m$, kemudian diikuti oleh $a$, $k$, $a$, dan $n$ sampai menemukan nilai 10. Proses pencarian bisa dilihat di Gambar ~\ref{TrieSearchingMakan}.

\begin{figure}
    \includegraphics[width=\textwidth,keepaspectratio]{fig/TrieSearchingMakan.png}%
	\caption{Mencari kata $makan$ di Trie}%
	\label{fig:TrieSearchingMakan}%
\end{figure}

Pencarian \textit{node} akan berakhir apabila tidak ada \textit{node} yang cocok dengan kata pencarian kita, atau menemukan kata yang kita cari. Akan tetapi, walaupun kita sudah menemukan kata yang dicari, apabila tidak ada nilai dalam \textit{node} tersebut maka itu akan dianggap tidak ketemu. 

Contohnya jika kita mencari kata $maka$, akan ditemukan di jalur $makan$ tetapi tidak ada ditemukan nilai (null) di akhir pencarian maka kata $maka$ akan dianggap tidak ada di dalam $trie$.

\begin{figure}
    \includegraphics[width=\textwidth,keepaspectratio]{fig/TrieSearchingMaka.png}%
	\caption{Kata $maka$ ditemukan tetapi tidak memiliki nilai (bernilai null).}%
	\label{fig:TrieSearchingMaka}%
\end{figure}

Contoh lain, jika kita mencari kata $makin$, maka kita akan berhenti di huruf $k$ dan tidak bisa meneruskan lagi sehingga dianggap kata $makin$ tidak terdapat di trie.

\begin{figure}
    \includegraphics[width=\textwidth,keepaspectratio]{fig/TrieSearchingMakin.png}%
	\caption{Pencarian berhenti di huruf $k$ dan tidak bisa diteruskan sehingga kata $makin$ tidak ditemukan di trie.}%
	\label{fig:TrieSearchingMakin}%
\end{figure}

Pembentukan \textit{trie} dilakukan dengan menambahkan setiap string ke \textit{trie} dengan langkah berikut.
\begin{enumerate}
	\item Menggunakan karakter pertama dari string sebagai panduan untuk memasukkan ke \textit{trie}. Jika karakter sudah ada di \textit{trie} berupa \textit{node} maka gunakan \textit{node} tersebut untuk merepresentasikan karakter tersebut. Lanjutkan dengan karakter selanjutnya dari string.
	\item Jika tidak ada karakter yang ingin kita masukkan di \textit{trie} maka bentuk node baru.
	\item Apabila semua karakter sudah dimasukkan ke trie, masukkan nilai (\textit{value}) dari string ke karakter terakhir.
\end{enumerate}

Sebagai contoh, kita akan membentuk sebuah \textit{trie} dengan menggunakan string berikut: $she, sells, sea, shells, by, the, sea, shore$.

\begin{enumerate}
	\item Pertama masukkan string $she$. Karena belum ada \textit{node} dengan karakter awal $s$ maka, kita akan bentuk \textit{node} baru untuk semua string $she$. Kemudian berikan nilai 0 di akhir string $she$.
		\begin{figure}
			\includegraphics[width=\textwidth,keepaspectratio]{fig/FormTrie1.png}%
			\caption{String $she$ dimasukkan ke dalam Trie.}%
			\label{fig:FormTrie1}%
		\end{figure}
	\item Setelah string $she$, maka selanjutnya adalah string $sells$. Dalam memasukkan string $sells$, karakter pertama yaitu $s$ sudah dimasukkan, untuk itu \textit{node} untuk $s$ yang digunakan oleh $she$ juga digunakan untuk $sells$. Akan tetapi, karakter selanjutnya yaitu $e$ tidak ada setelah $s$ maka itu dibentuk node baru untuk $e$. Kemudian berikan nilai 1 di akhir $sells$.
		\begin{figure}
			\includegraphics[width=\textwidth,keepaspectratio]{fig/FormTrie2.png}%
			\caption{String $sells$ dimasukkan ke dalam Trie.}%
			\label{fig:FormTrie2}%
		\end{figure}
	\item Lakukan hal yang sama untuk semua string yang tersisa.
		\begin{figure}
			\includegraphics[width=\textwidth,keepaspectratio]{fig/FormTrie3.png}%
			\caption{Semua string dimasukkan ke dalam Trie.}%
			\label{fig:FormTrie3}%
		\end{figure}
\end{enumerate}

\subsection{Implementasi Python}

Untuk implementasi di python, akan menggunakan \textit{dictionary} sebagai dasar dari \textit{Trie}. Kita akan membuat sebuah \textit{class} \textit{Trie} yang terdapat beberapa metode seperti penambahan, penghapusan, dan sebagainya.

Kita akan melihat inisialisasi dari kelas \textit{Trie} yang dapat dilihat di Algoritma ~\ref{algo:trieinit}.
Dalam inisialisasi Trie tersebut variabel $path$ merupakan \textit{dictionary} yang akan menyimpan semua \textit{node} dari Trie. Sedangkan variabel $value$ akan menyimpan nilai dari \textit{node} tersebut. Variabel $value\_valid$ akan menyimpan data boolean yang menandakan apakah \textit{node} tersebut valid atau tidak.

\lstinputlisting[language=Python, 
                 firstline=1,
                 lastline=6,
                 label={algo:trieinit},
                 caption=Inisialisasi Kelas Trie
                ]
                {code/trie.py}
								
Untuk penambahan string baru bisa menggunakan metode di Algoritma ~\ref{algo:trieAdd}. 

\lstinputlisting[language=Python, 
                 firstline=8,
                 lastline=21,
                 label={algo:trieAdd},
                 caption=Penambahan String baru ke Trie
                ]
                {code/trie.py}

Dalam Algoritma ~\ref{algo:trieAdd}, pertama karakter awal dari string akan dimasukkan ke dalam variabel $head$. 

\lstinputlisting[language=Python, 
                 firstline=9,
                 lastline=9,        
                ]
                {code/trie.py}
								
Kemudian, kita akan mengecek apakah sebelumnya sudah ada karakter tersebut di dalam \textit{Trie}. Jika ada maka kita akan mulai dari \textit{node} yang berisikan karakter tersebut. Jika tidak, maka kita akan bentuk \textit{node} baru.

\lstinputlisting[language=Python, 
                 firstline=10,
                 lastline=14,        
                ]
                {code/trie.py}

Setelah itu kita cek apakah karakter yang kita masukkan merupakan yang terakhir dari string. Jika merupakan karakter terakhir maka kita akan akhiri dengan menset \textit{value} dari \textit{node} tersebut. Jika masih tersisa karakter lain, maka kita akan secara rekursif untuk memproses karakter lainnya.

\lstinputlisting[language=Python, 
                 firstline=16,
                 lastline=21,        
                ]
                {code/trie.py}

Untuk penghapusan, kita akan menggunakan metode di ~\ref{algo:trieremove}. 

\lstinputlisting[language=Python, 
                 firstline=23,
                 lastline=34,
                 label={algo:trieremove},
                 caption=Penghapusan node di Trie
                ]
                {code/trie.py}
								
Seperti penambahan, penghapusan string akan dimulai dengan pengecekan karakter pertama apakah ada di dalam \textit{trie} atau tidak.

\lstinputlisting[language=Python, 
                 firstline=24,
                 lastline=25,        
                ]
                {code/trie.py}

Jika ada dalam \textit{trie} maka, kita bisa menghapus string tersebut dengan menggunakan rekursif. 

\lstinputlisting[language=Python, 
                 firstline=26,
                 lastline=34,        
                ]
                {code/trie.py}

Untuk mengambil \textit{value} dari sebuah string, kita bisa menggunakan metode di ~\ref{algo:Trieget} dimana proses pengambilan \textit{value} dilakukan dengan rekursif sampai karakter terakhir untuk mengambil nilainya.

\lstinputlisting[language=Python, 
                 firstline=36,
                 lastline=51,
                 label={algo:Trieget},
                 caption=Mengambil \textit{value} node di Trie
                ]
                {code/trie.py}

Untuk mengecek apakah sebuah string ada atau tidak di dalam Trie bisa menggunakan metode di Algoritma ~\ref{algo:Triecontain}

\lstinputlisting[language=Python, 
                 firstline=36,
                 lastline=51,
                 label={algo:Triecontain},
                 caption=Mengecek apakah sebuah string ada atau tidak di Trie
                ]
                {code/trie.py}
								
Untuk mengambil semua string di Trie dengan prefix tertentu atau semuanya bisa menggunakan metode di Algoritma ~\ref{algo:Triekeys}.

\lstinputlisting[language=Python, 
                 firstline=36,
                 lastline=51,
                 label={algo:Triekeys},
                 caption=Mengambil semua string di Trie
                ]
                {code/trie.py}

\subsection{\textit{Suffix Trie}}

\textit{Suffix Trie} adalah sebuah \textit{Trie} biasa dimana isi dari \textit{Trie} tersebut adalah semua \textit{suffix} dari sebuah string. Sebagai contoh untuk string $abaaba$, maka kita bisa membentuk suffix berupa: $abaaba\$$, $baaba\$$, $aaba\$$, $aba\$$, $ba\$$, $a\$$, dan $\$$. Semua itu akan dibentuk sebuah \textit{trie} yang bisa dilihat di Gambar ~\ref{fig:suffixTrie}

	\begin{figure}
		\includegraphics[width=\textwidth,keepaspectratio]{fig/suffixTrie.png}%
		\caption{Sebuah \textit{suffix trie} dari string $abaaba$.}%
		\label{fig:suffixTrie}%
	\end{figure}

Implementasi dari \textit{Suffix Trie} sama seperti \textit{Trie} biasa hanya saja kita harus mencari dulu semua suffix dari sebuah string baru dimasukkan ke dalam \textit{Trie}. Untuk mencari semua suffix cukup iterasi dari awal string sampai akhir string dan potong satu karakter dari depan setiap kali iterasi.

Dengan menggunakan \textit{suffix trie}, banyak operasi pada string bisa dilakukan dengan cepat. \textit{Suffix trie} lebih efisien dalam penyimpanan dibandingkan dengan menyimpan dalam array biasa. Salah satu contoh pemanfaatan \textit{Suffix Trie} adalah mencari apakah sebuah substring merupakan bagian dari sebuah string misalnya: apakah substring $baa$ merupakan bagian dari $abaaba$? Untuk penyelesaiannya bisa dilihat di Gambar ~\ref{fig:suffixtriefindbaa}. Waktu yang diperlukan untuk mencari adalah sepanjang ukuran query atau O(|query|).

	\begin{figure}
		\includegraphics[width=\textwidth,keepaspectratio]{fig/suffixtriefindbaa.png}%
		\caption{Mencari substring $baa$ di string $abaaba$.}%
		\label{fig:suffixtriefindbaa}%
	\end{figure}

Ada banyak apalikasi lain untuk \textit{suffix trie} seperti:
\begin{enumerate}
	\item Mencari apakah sebuah string merupakan suffix dari string lain.
	\item Mencari \# kemunculan dari sebuah string dalam string lain.
	\item Mencari jumlah substring yang paling banyak muncul di sebuah string.
\end{enumerate}

\subsection{\textit{Suffix Tree}}

\textit{Suffix Tree} adalah pengembangan dari \textit{Suffix Trie} dimana semua \textit{node} yang tidak memiliki cabang dikompress menjadi satu \textit{node}. Lihat Gambar ~\ref{fig:suffixtree} untuk melihat \textit{suffix tree} dari string $abaaba$.

	\begin{figure}
		\includegraphics[width=\textwidth,keepaspectratio]{fig/suffixtree.png}%
		\caption{Suffix tree dari string $abaaba$. Perhatikan bahwa pohon menjadi lebih kecil dan lebih ruang efisien}%
		\label{fig:suffixtree}%
	\end{figure}
	
\subsection{\textit{Suffix Array}}

\textit{Suffix Array} merupakan \textit{Suffix Tree} yang disimpan di dalam sebuah array. Semua suffix dari sebuah string disimpan dalam array. Posisi setiap suffix dalam array sesuai dengan urutan \textit{lexicographic} (diurut berdasarkan alphabet setelah diurut berdasarkan panjang suffix). Lihat Gambar ~\ref{fig:suffixarray} untuk melihat \textit{Suffix Array} dari string $abaaba$.

	\begin{figure}
		\includegraphics[width=\textwidth,keepaspectratio]{fig/suffixarray.png}%
		\caption{Suffix array dari string $abaaba$.}%
		\label{fig:suffixarray}%
	\end{figure}
	
Implementasi \textit{Suffix Array} adalah yang termudah dari semua. Lihat Algoritma ~\ref{algo:SuffixArray} untuk implementasi \textit{suffix array}.

\lstinputlisting[language=Python, 
                 firstline=4,
                 lastline=9,
                 label={algo:SuffixArray},
                 caption=Membentuk suffix array
                ]
                {code/suffixArray.py}

Untuk \textit{Suffix Array}, kita akan menggunakan \textit{OrderedDict} yang merupakan bagian dari \textit{collections}. Dengan \textit{dictionary} biasa kita tidak bisa mengurut isi dari \textit{dictionary}, akan tetapi dengan \textit{OrderedDict} kita bisa mengurut berdasarkan kunci. 

\section{Algoritma Pemrosesan String}

Untuk pemrosesan String terdapat banyak algoritma, kita akan melihat beberapa contoh algoritma yang bisa memanfaatkan \textit{suffix array}. Kita memilih menggunakan \textit{suffix array} dikarenakan implementasi \textit{suffix array} adalah yang paling mudah dibandingkan dengan \textit{suffix trie} dan \textit{suffix tree}.

\subsection{Pencarian \textit{pattern} Substring dalam String (\textit{String Matching})}

Dengan menggunakan \textit{Suffix Array} dari string $T$, kita bisa mencari \textit{pattern} string $P$ (dengan panjang $m$) dari string $T$ (panjang $n$) tersebut. Pencarian dengan menggunakan \textit{suffix array} lebih mudah dikarenakan 2 hal berikut.

\begin{enumerate}
	\item Jika seandainya $P$ adalah substring dari $T$, maka $P$ harus merupakan \textit{prefix} dari salah satu {suffix} $T$. 
	\item Karena dalam \textit{suffix array}, semua \textit{suffix} sudah diurut berdasarkan \textit{lexicography} maka pencarian \textit{prefix} lebih sederhana. Kita cukup mencocokkan setiap huruf dari urutan pertama dalam \textit{suffix array} sampai urutan terakhir secara berturut. 
\end{enumerate}

Cara naif untuk mencari \textit{pattern} substring dalam sebuah string dalam \textit{suffix array} adalah menggunakan pencarian sekuensial untuk mencocokkan \textit{prefix} dari semua \textit{suffix} yang ada dalam \textit{suffix array}. Untuk illustrasi bisa dilihat di Gambar ~\ref{fig:searchABSuffixArraySequential}.

	\begin{figure}
		\includegraphics{fig/searchABSuffixArraySequential.png}%
		\caption{Mencari \textit{pattern} $ab$ dari string $abaaba$}%
		\label{fig:searchABSuffixArraySequential}%
	\end{figure}

Untuk implementasi pada Python bisa dilihat di Algoritma ~\ref{algo:StringMatchingNaif}.

\lstinputlisting[language=Python, 
                 firstline=11,
                 lastline=16,
                 label={algo:StringMatchingNaif},
                 caption=\textit{String Matching} naif dengan suffix array
                ]
                {code/suffixArray.py}


Untuk mempercepat pencarian, daripada menggunakan pencarian sekuensial kita akan menggunakan dua kali \textit{Binary Search} untuk mencari di bagian batas bawah dan batas atas. Waktu yang dibutuhkan untuk mencari adalah $O(m log n)$. Implementasi dari penggunaan \textit{Binary Search} bisa dilihat di Algoritma ~\ref{algo:StringMatchingBS}. 

\lstinputlisting[language=Python, 
                 firstline=18,
                 lastline=47,
                 label={algo:StringMatchingBS},
                 caption=\textit{String Matching} suffix array dengan Binary Search
                ]
                {code/suffixArray.py}

Cara kerja dari Algoritma ~\ref{algo:StringMatchingBS} bisa dilihat dari Gambar \ref{fig:StringMatchingBS} dimana kita akan mencari substring $GA$ dalama $GATAGACA$.

	\begin{figure}
		\includegraphics[width=\textwidth,keepaspectratio]{fig/StringMatchingBS.png}%
		\caption{\textit{String matching} dengan \textit{Binary Search}.}%
		\label{fig:StringMatchingBS}%
	\end{figure}
	
Perlu diketahui, bahwa berbeda dengan menggunakan pencarian sekuensial, kita tak perlu mencocokkan satu per satu setiap suffix dalam string $GATAGACA$. Melainkan kita hanya perlu mencari suffix pertama dan suffix terakhir dari string tersebut yang memuat substring $GA$. Sebagai contohnya, dari gambar kita mendapatkan suffix pertama yang memuat $GA$ adalah suffix posisi ke 6 (i=6) yaitu $GATAGACA$, dan suffix terakhir adalah suffix posisi ke 7 (i=7) yaitu $GACA$, maka secara otomatis semua suffix dalam posisi ke 6 dan ke 7 adalah suffix yang memuat $GA$ (karena posisi berdekatan maka hanya ada dua, coba lakukan sekali lagi dengan pencarian substring $A$).

Untuk mencari suffix pertama dan terakhir dibutuhkan dua kali \textit{Binary Search}. Satu untuk mencari yang pertama (\textit{lower bound}) dan satu lagi untuk mencari yang terakhir (\textit{upper bound}).

Untuk mencari suffix pertama kita akan menggunakan \textit{Binary Search} yang bisa dilihat di Algoritma ~\ref{algo:StringMatchingBS1}.

\lstinputlisting[language=Python, 
                 firstline=21,
                 lastline=29,
                 label={algo:StringMatchingBS1},
                 caption=\textit{Binary Search I} untuk mencari suffix pertama yang memuat substring
                ]
                {code/suffixArray.py}

Setelah kita menggunakan \textit{Binary Search} untuk mencari suffix pertama, kita akan mengecek apakah di posisi tersebut ada terdapat substring yang kita inginkan tidak. Jika tidak kita akan mengembalikan -1. Untuk pengecekan lihat di Algoritma ~\ref{algo:StringMatchingNF} 

\lstinputlisting[language=Python, 
                 firstline=30,
                 lastline=31,
                 label={algo:StringMatchingNF},
                 caption=Mengecek apakah di posisi tersebut terdapat substring yang kita inginkan atau tidak.
                ]
                {code/suffixArray.py}

Untuk mencari suffix kedua kita juga akan menggunakan \textit{Binary Search} yang bisa dilihat di Algoritma ~\ref{algo:StringMatchingBS2}.

\lstinputlisting[language=Python, 
                 firstline=21,
                 lastline=29,
                 label={algo:StringMatchingBS2},
                 caption=\textit{Binary Search II} untuk mencari suffix kedua yang memuat substring
                ]
                {code/suffixArray.py}



\subsection{Pencarian pattern terpanjang yang berulang di dalam string}

Diberikan sebuah string $T$, kita ingin mencari pattern $P$ terpanjang yang berulang (minimal >1) dalam string $T$. Misalnya: dalam string $banana$, maka pattern terpanjang yang berulang adalah $ana$.

Untuk menyelesaikan masalah tersebut, kita memerlukan sebuah struktur data yang lebih efisien dan merupakan pengembangan dari \textit{suffix array} yaitu \textit{longest common prefix suffix array} (LCPA. Dalam LCPA, semua panjang LCP akan dihitung dan disimpan dalam \textit{suffix array}. Definisi dari LCP adalah: ``Diberikan dua buah string yaitu $u$ dan $v$, maka prefix terpanjang yang terdapat di kedua buah string tersebut adalah LCP''. Sebagai contohnya:

LCP dari string $\textbf{a}$ dan $\textbf{a}abba$ adalah $a$ dengan panjang 1.

LCP dari string $\textbf{ab}aaabba$ dan $\textbf{ab}ba$ adalah $ab$ dengan panjang 2.

Sebagai contoh dari LCPA, kita bisa melihat contoh berikut dari string $banana$.


\begin{center}
  \begin{tabular}{ | c | c | c |c | c | c | c | c | }
    \hline
    i	& 1	& 2	& 3	& 4	& 5	& 6	& 7 \\ \hline
		A[i]	& 7	& 6	& 4	& 2	& 1	& 5	& 3 \\ \hline
		H[i]	& null	& 0 & 1 &	3 & 0 & 0 & 2 \\ \hline
		1	& \$	& a	& a	& a	& b	& n	& n \\ \hline
		2	& & \$	& n	& n	& a	& a	& a \\ \hline
		3	& & & a	& a	& n	& \$	& n \\ \hline
		4	& & & \$	& n	& a	& & a \\ \hline
		5 & & & & a	& n	& & \$ \\ \hline
		6	& & & & \$ & a	& & \\ \hline
		7	& & & & & \$ & &  \\ 
    \hline
  \end{tabular}
\end{center}

Dari tabel, H[i] merupakan LCP dari setiap suffix $banana$ yang ada. Cara membaca tabel adalah kita membanding setiap H[i] dengan H[i-1] dan mencari jumlah prefix terpanjang. Sebagai contohnya H[3] adalah 1 karena H[2] ($a\$$) dan H[3]($ana\$$) ada kesamaan prefix dengan panjang sebanyak 1 ($a$). Sedangkan H[4] adalah 3 karena H[4] ($anana\$$) dan H[3] ($ana\$$) memiliki kesamaan prefix dengan panjang sebanyak 3 ($ana$). Dengan menyusun LCPA dari setiap suffix, kita bisa mendapatkan prefix dengan jumlah karakter terpanjang sebanyak 3 yang berada di suffix ($anana\$$).

Algoritma pembentukan LCPA bisa dilihat di ~\ref{algo:LCPA}. Algoritma tersebut membentuk LCPA dengan menghitung prefix yang sama dari setiap suffix.

\lstinputlisting[language=Python, 
                 firstline=49,
                 lastline=58,
                 label={algo:LCPA},
                 caption=Pembentukan LCPA dari \textit{suffix array}
                ]
                {code/suffixArray.py}

%\documentclass[12pt]{book}%ditambakan
\usepackage{graphicx}%ditambahkan
\usepackage{placeins}%ditambahkan
\usepackage{algpseudocode}%ditambahkan
\usepackage{listings}%ditambahkan
\usepackage{amsmath}%ditambahkan

\begin{document}%ditambahkan
\chapter{Teknik \textit{Divide and Conquer}}\label{ch:modul4}

\section{Pengenalan \textit{Divide and Conquer}}
\textit{Divide and Conquer} merupakan salah satu pendekatan dalam menyelesaikan permasalahan algoritma. \textit{Divide and Conquer} terdiri atas tiga langkah:
\begin{enumerate}
	\item \textbf{\textit{Divide}/Memecah} --- Sebuah permasalahan yang ada dibagi menjadi permasalahan yang lebih kecil (\textit{subproblem}).
	\item \textbf{\textit{Conquer}/Menyelesaikan} --- Menyelesaikan setiap \textit{subproblem} yang ada secara rekursif. Jika \textit{subproblem} tersebut sangat kecil atau tak bisa dipisahkan lagi, maka diselesaikan secara langsung dengan menggunakan algoritma yang sederhana.
	\item \textbf{\textit{Combine}/Menggabungkan} --- Menggabungkan setiap solusi dari \textit{subproblem} yang telah diselesaikan menjadi sebuah solusi yang lengkap dan optimal.
\end{enumerate}

Teknik \textit{Divide and Conquer} digambarkan dalam Figur \ref{fig:DivideandConquerIllustration}, yang menunjukkan pembagian sebuah masalah menjadi dua submasalah yang lebih sederhana.

\newpage{}
\begin{figure}[htbp]
\begin{center}
	\includegraphics[scale=0.7]{Ilustrasi.jpg}%
	\caption{Illustrasi dari Divide and Conquer}%
	\label{fig:DivideandConquerIllustration}%
\end{center}
\end{figure}

Sebagai Contoh, untuk menghitung total jumlah dari element-element dalam List, kita dapat menggunakan perulangan sederhana:



\lstset{language=Python}
\label{lst:SimpleSum}
\begin{lstlisting}[frame=single]
A = [4,1,3,5,6,7,2]
total = 0
  for i in range(0,len(A)):
    total = total + A[i]
print(total)
\end{lstlisting}

Algoritma perulangan yang digunakan pada kode di atas dapat memberikan hasil yang benar, tetapi terdapat beberapa masalah pada kode tersebut, yaitu perhitungan dilakukan secara linear, yang menghasilkan kompleksitas $O(n)$. Hal ini tentunya cukup ideal untuk ukuran list kecil, tetapi jika ukuran list menjadi besar (beberapa Milyar elemen) maka perhitungan akan menjadi sangat lambat.

Dengan menerapkan Teknik \textit{Divide and Conquer}, maka masalah penjumlahan sebelumnya dapat dibagi menjadi submasalah yang lebih kecil. Langkah pertama yang dapat dilakukan adalah menerapkan teknik rekursif untuk membagi-bagikan masalah menjadi masalah yang lebih kecil. Jika awalnya kita harus menghitung total keseluruhan list satu per satu, sekarang kita dapat melakukan perhitungan dengan memecah-mecah list terlebih dahulu:

\lstset{language=Python}
\label{lst:DivideAndConquerSum}
\begin{lstlisting}[frame=single]
def SumOfList(MyList):
  if len(MyList) >= 1:
    return MyList[0] 
	mid = len(MyList) // 2
  left = SumOfList(MyList[:mid])
  right = SumOfList(MyList[mid:])
  return left + right
	
\end{lstlisting}

Setelah membagikan list menjadi dua bagian terus menerus sampai bagian terkecilnya, penjumlahkan kedua nilai list tersebut dapat dilihat pada Figur \ref{fig:DivideAndConquerSum} berikut:

\begin{figure}[htbp]
\begin{center}
	\includegraphics[scale=0.7]{Sum.png}%
	\caption{Illustrasi Penjumlahan Divide and Conquer}%
	\label{fig:DivideAndConquerSum}%
\end{center}
\end{figure}

Algoritma \textit{Divide and Conquer}, jika diimplementasikan menggunakan library atau bahasa yang tepat akan meningkatkan efisiensi algoritma secara logaritmik. Untuk melihat kompleksitas algoritma perhatikan analisis pada fungsi SumOfList dengan teknik \textit{Divide and Conquer} berikut ini:

\lstset{language=Python}
\label{lst:DivideAndConquerSum}
\begin{lstlisting}[frame=single]
def SumOfList(MyList):
  if len(MyList) >= 1:		#cost = a 
    return MyList[0]		#cost = b 
	mid = len(MyList) // 2	#cost = c
  left = SumOfList(MyList[:mid]) #cost = f(n/2) + h
  right = SumOfList(MyList[mid:]) #cost = f(n/2) + i
  return left + right	#cost = d
	
\end{lstlisting}

yang secara matematis dapat dituliskan seperti berikut:
$$	  \mathnormal{f(n) = a+b+c+f(\frac{n}{2})+h+f(\frac{n}{2})+i+d } $$
$$	  \mathnormal{f(n) = 2f(\frac{n}{2})} $$

karena ukuran dari mid adalah panjang list $(n)$ dibagi dua. Dengan begitu, kompleksitas dari algoritma adalah:

\begin{figure}[htbp]
\begin{center}
	\includegraphics[scale=0.8]{fn.jpg}%
	\caption{Illustrasi pembagian list}%
	\label{fig:PembagianList}%
\end{center}
\end{figure}

dengan syarat berhenti adalah $k \geq 1$ sehingga:

$$	  \mathnormal{ \frac{n}{2^{k}}=1} $$
$$	  \mathnormal{n=2^{k}} $$
$$	  \mathnormal{k=log_{2}n} $$

Dibandingkan dengan Algoritma penjumlahan dengan perulangan $(O(n))$, Kompleksitas dari fungsi penjumlahan dengan teknik \textit{Divide and Conquer} diatas adalah $O(log n)$.

\section{\textit{Binary Search}}
Algoritma \textit{Binary search}  merupakan  algoritma untuk melakukan pencarian pada list yang sudah mempunyai syarat bahwa list sudah terurut, baik secara menaik ataupun menurun. Algoritma \textit{Binary search} menggunakan 3 langkah dari \textit{Divide and Conquer} sebagai berikut:
\begin{enumerate}
\item \textbf{\textit{Divide}} --- Membagi $n$ elemen/bilangan menjadi dua bagian dimana masing-masing adalah $n/2$. Setiap $n$ bilangan akan terus dibagi dua menjadi dua bagian kiri dan kanan hingga data yang dicari ditemukan.
\item \textbf{\textit{Conquer}} --- Cari nilai kunci dalam  List terurut dan mengembalikan indeks di mana kunci itu ditemukan atau -1 jika tidak ditemukan dan membagi List secara rekursif.
\item \textbf{\textit{Combine}} --- Tahap ini tidak perlu dilakukan karena yang diperlukan hanya 1 element berupa data yang dicari.
\end{enumerate}

 Misalnya untuk menemukan nilai dari \textit{Val}  dalam List \textit{MyList} dengan menggunakan Algoritma \textit{Binary search}, maka ada 3 kemungkinan kondisi pada \textit{Binary search} yaitu:
\begin{enumerate}
  \item Jika nilai dari \textit{Val} di temukan pada MyList[middle], maka proses pembagian ruangan berhenti. 
  \item Jika nilai dari \textit{Val} $<$ MyList[middle], maka pencarian di batasi hanya pada bagian sebelah kiri dari List. Seluruh elemen yang berada di sebelah kanan dapat di abaikan.
  \item Jika nilai dari \textit{Val} $>$ MyList[middle], maka pencarian di batasi hanya pada bagian sebelah kanan dari List. Seluruh elemen yang berada di sebelah kiri dapat di abaikan.
  \item Jika seluruh data telah di cari namun nilai dari \textit{Val} tidak ditemukan, maka diberi nilai seperti -1.
\end{enumerate}

Sebagai Contoh, pencarian nilai 15 dari sebuah list dapat dilihat pada Figur \ref{fig:Binary Search Ilustration} berikut:

\begin{figure}[htbp]
\begin{center}
	\includegraphics[scale=0.8]{BinarySearch.jpg}%
	\caption{Illustrasi Binary Search}%
	\label{fig:Binary Search Ilustration}%
\end{center}
\end{figure}

\newpage{}
Algoritma \textit{Binary search} dapat diterapkan dengan menggunakan cara rekursif maupun nonrekursif. Untuk cara rekursif dapat diimplementasikan sebagai berikut:

\lstset{language=Python}
\label{lst:BinarySearch}
\begin{lstlisting}[frame=single]
def BinarySearch(MyList, val, left, right):
  if right < left:
    print("%d not found in list"%val)
    return -1
	
  mid = (left + right) // 2
  if MyList[mid] > val:
    return BinarySearch(MyList, val, left, right - 1)
  elif data[mid] < val:
    return binary_search(MyList, val, left + 1, right)
  else:
    print("%d found in index %d"%(val, mid))
  
  return mid
\end{lstlisting}

\subsection{Analisi Binary Search}
Untuk analisis metode binary search dalam menentukan kompleksitasnya, perhatikan untuk sebuah array, misalkan dengan panjang $(n) = 8$, maka :\newline
Ketika n=8, Binary Search akan mereduksi ukuran listnya menjadi 4\newline
Ketika n=4, Binary Search akan mereduksi ukuran listnya menjadi 2\newline
Ketika n=2, Binary Search akan mereduksi ukuran listnya menjadi 1\newline

Dapat dilihat bahwa binary search dipanggil sebanyak tiga kali untuk $n = 8$. Sehingga didapat $8 = 2^{3}$ atau secara general dapat dikatakan $n = 2^{k}$.  Nilai dari k dapat dinotasikan menjadi $2^{k} = n$ sehingga $k = log_2\ n$ sehingga didapatkan untuk kempleksitas dari binary search adalah  O (log n).

Misalnya jika kita memiliki array sebanyak $n$ elemen, maka kasus tersebut melakukan pembandingan list sebanyak $log\ n$. Bandingkan dengan linear search yang  melakukan pembandingan sebanyak $n$. Tentunya algoritma binary search menjadi lebih cepat. Namun jika data yang ada merupakan data yang tidak terurut maka akan jauh lebih cepat jika menggunakan linear search sehingga penggunaan binary search akan ada cost tambahan yaitu dalam mengurutkan List.

\section{\textit{Merge Sort}}
Algoritma \textit{Merge Sort} merupakan algoritma pengurutan yang mengikuti pendekatan \textit{Divide and Conquer}. Algoritma tersebut menggunakan 3 langkah dari \textit{Divide and Conquer} sebagai berikut:
\begin{enumerate}
\item \textbf{\textit{Divide}} --- Membagi $n$ elemen/bilangan menjadi dua bagian dimana masing-masing adalah $n/2$. Setiap $n$ bilangan akan terus dibagi dua sampai habis atau bilangan tinggal 1 saja.
\item \textbf{\textit{Conquer}} --- Mengurut setiap bagian secara rekursif menggunakan \textit{merge sort}. 
\item \textbf{\textit{Combine}} --- Menggabungkan dua bagian yang telah diurut untuk menghasilkan satu bagian keseluruhan yang sudah terurut.
\end{enumerate}

Figur \ref{fig:mergeSortIllustration} menggambarkan cara kerja Merge Sort terhadap 4 buah bilangan yang tak terurut. Di illustrasi tersebut \textit{List} yang memiliki 4 bilangan tak terurut dipecah-pecah sampai menjadi 1 buah bilangan atau panjang \textit{List} 1 ($A.length = 1$). Karena satu buah bilangan sudah merupakan bilangan yang terurut maka pecahan bilangan tersebut digabungkan sampai menjadi satu \textit{List} bilangan yang terurut. 

\begin{figure}[htbp]
\begin{center}
	\includegraphics[scale=0.8]{mergeSort1.jpg}%
	\caption{Illustrasi dari cara kerja merge sort}%
	\label{fig:mergeSortIllustration}%
\end{center}
\end{figure}


Algoritma \textit{merge sort} terdiri dari dua algoritma utama yaitu Algoritma (MERGE($left,right$)) dan Algoritma (MERGE-SORT($MyList$)). 

Algoritma MERGE-SORT($MyList$) ditujukan untuk membagi (\textit{divide}) sebuah \textit{List} $MyList$ menjadi dua bagian secara rekursif sampai \textit{List} tersebut tak bisa dibagi lagi (panjang \textit{List} $MyList$ adalah 1). 

\lstset{language=Python}
\label{lst:MargeSort}
\begin{lstlisting}[frame=single]
def merge_sort(MyList):
  if len(MyList) <= 1:
    return MyList
  mid = len(MyList) // 2
  left = merge_sort(MyList[:mid])
  right = merge_sort(MyList[mid:])
	
  return merge(left, right)
\end{lstlisting}

Algoritma MERGE($left,right$) ditujukan untuk menggabungkan (\textit{combine}) kumpulan \textit{List} yang sudah diurut (memiliki panjang 1) menjadi sebuah \textit{List} baru yang sudah terurut.

\lstset{language=Python}
\label{lst:Marge}
\begin{lstlisting}[frame=single]
def merge(left, right):
  result = []
	
  while len(left) > 0 or len(right) > 0:
    if len(left) > 0 and len(right) > 0:
      if left[0] <= right[0]:
         result.append(left.pop(0))
      else:
         result.append(right.pop(0))
    elif len(left) > 0:
       result.append(left.pop(0))
    elif len(right) > 0:
       result.append(right.pop(0))

    return result
\end{lstlisting}

Perlu diketahui di algoritma \textit{merge sort} tidak diperlukan fungsi khusus untuk \textit{Conquer} dikarenakan \textit{List} yang memiliki panjang 1 secara otomatis sudah terurut.

\newpage{}
\subsection{Analisi Marge Sort}
Seberapa efisienkah \textit{merge sort}? Berikut analisi dari kompleksitas \textit{merge sort}.

\lstset{language=Python}
\label{lst:MargeSort}
\begin{lstlisting}[frame=single]
def merge_sort(MyList):
  if len(MyList) <= 1:	#cost = a
    return MyList		#cost = b
  mid = len(MyList) // 2	#cost = c
  left = merge_sort(MyList[:mid]) #cost = f(n/2) + h
  right = merge_sort(MyList[mid:]) #cost = f(n/2) + i
	
  return merge(left, right)	#cost = g.n+j
\end{lstlisting}

yang secara matematis dapat dituliskan seperti berikut:
$$	  \mathnormal{f(n) = a+b+c+f(\frac{n}{2})+h+f(\frac{n}{2})+i+g.n+j } $$
$$	  \mathnormal{f(n) = 2f(\frac{n}{2})+a.n + b} $$

dengan syarat $n > 1$, maka:
$$	  \mathnormal{f(n) = 2f(\frac{n}{2})+a.n } $$
$$	  \mathnormal{f(n) = 2(2f(\frac{n}{4}))+a.\frac{n}{2} } + a+n $$
$$	  \mathnormal{f(n) = 4f(\frac{n}{4})+2a.n } $$
$$	  \mathnormal{f(n) = 2(4f(\frac{n}{8}))+a.\frac{n}{4} }+2a.n $$
$$	  \mathnormal{f(n) = 8f(\frac{n}{8})+3a.n } $$
$$	  \mathnormal{f(n) = 16f(\frac{n}{16})+4a.n } $$
$$	  \mathnormal{f(n) = 2^{k}f(\frac{n}{2^{k}})+k.a.n } $$

untuk $ \mathnormal{f(n) = (\frac{n}{2^{k}})} $ :

$$ \mathnormal{\frac{n}{2^{k}}} $$
$$ \mathnormal{2^{k}=n} $$
$$ \mathnormal{k=log_{2}n} $$

sehingga :

$$	  \mathnormal{f(n) = 2^{log_{2}n}f(1)+log_{2}n.a.n } $$

$$	  \mathnormal{f(n) = a.n\ log\ n = O(n\ log\ n)} $$

MargeSort mempunyai kelemahan utama yaitu jumlah lokasi penyimpanan ekstra yang dibutuhkan oleh algoritma.
 
\section{\textit{Quick Sort}}

\textit{Quick sort} seperti \textit{merge sort} juga menggunakan pendekatan \textit{Divide and Conquer}. Tiga langkah yang dimiliki \textit{quick sort} ialah:
\begin{enumerate}
	\item \textit{Divide} --- Mempartisi (menyusun) \textit{List} $MyList[p..r]$ menjadi dua \textit{subarray} yang berukuran lebih kecil yaitu $MyList[l..q-1]$ dan $MyList[q+1..r]$ sehingga setiap elemen dari $MyList[l..q-1]$ lebih kecil atau sama dengan $A[q]$ yang mana akan lebih kecil atau sama dari setiap elemen $A[q+1..r]$. Indeks $q$ dihitung melalui fungsi partisi.
	\item \textit{Conquer} --- Urut dua \textit{subarray} $A[p..q-1]$ dan $A[q+1..r]$ secara rekursif.
	\item \textit{Combine} --- Tahap ini tidak perlu dilakukan karena sudah tergabung secara otomatis.
\end{enumerate}

Teknik \textit{Quick sort} berbeda teknik \textit{merge sort} dimana \textit{merge sort} nenbagi elemen-elemen dalam list berdasarkan posisi, \textit{quick sort} membagi elemen-elemen dalam List tersebut berdasrkan nilai. Secara spesifik, \textit{quick sort} mengatur kembali setiap elemen yang ada dalam list berdasarkan \emph{partisinya}. Sebuah variasi dari algoritma \textit{quick sort} adalah algoritma \textit{randomized-quick sort} dimana di algoritma tersebut memilih \textit{partisi} secara acak (\textit{random}). Hal tersebut dilakukan dengan harapan pemotongan rangkaian (\textit{partition}) lebih terdistribusi secara merata.

\newpage{}
Adapun ilustrasi dari cara kerja dari \textit{quick sort} dapat dilihat pada Figur \ref{fig:Quick Sort Ilustration} berikut:

\begin{figure}[htbp]
\begin{center}
	\includegraphics[scale=0.8]{QuickSort.jpg}%
	\caption{Ilustrasi Quick Sort}%
	\label{fig:Quick Sort Ilustration}%
\end{center}
\end{figure}

Kerugian dari versi sederhana diatas adalah membutuhkan ruang simpan yang lebih besar, yang sama buruknya seperti merge sort. Memory tambahan yang dibutuhkan dapat juga secara radikal berpengaruh pada kecepatan dan performa cache pada implementasi praktiknya. Terdapat juga versi yang lebih rumit yang menggunakan algoritma partisi \textit{in-place}. Ilustrasi dari algoritma \textit{quick sort} dengan partisi \textit{in-place}. Algoritma partisi \textit{in-place} digunakan untuk memisahkan bagian dari list antara index kiri dan kanan, dengan memindahkan seluruh elemen kurang dari MyList[pivotIndex] sebelum pivot, dan elemen yang sama atau lebih besar darinya. Dalam prosesnya, algoritma ini juga mencari posisi akhir untuk elemen pivot kembali. 


Adapun ilustrasi dari cara kerja dari \textit{quick sort} dengan partisi \textit{in-place} dapat dilihat pada Figur \ref{fig:Quick Sort Ilustration 2} berikut:

\begin{figure}[htbp]
\begin{center}
	\includegraphics[scale=0.5]{QuickSort2.jpg}%
	\caption{Ilustrasi Quick Sort In-Place}%
	\label{fig:Quick Sort Ilustration 2}%
\end{center}
\end{figure}

\newpage{}
Dalam Python Algoritma \textit{quick sort} dapat diterapkan sebagai berikut:
\lstset{language=Python}
\label{lst:QuickSort}
\begin{lstlisting}[frame=single]
def quickSort(myList, start, end):
 if start < end:
   pivot = partition(myList, start, end)
   quicksort(myList, start, pivot-1)
   quicksort(myList, pivot+1, end)
   return myList
\end{lstlisting}

\lstset{language=Python}
\label{lst:Partition}
\begin{lstlisting}[frame=single]
def partition(myList,start,end):
  pIdx = random.randrange(start,end)
  pivot = myList[pIdx]
  myList[pIdx],myList[end] = myList[end],myList[pIdx]
  mid = start
  for i in range(start,end-1):
   if(myList[i] < pivot):    
	 myList[i],myList[mid] = myList[mid],myList[i]
	 mid = mid+ 1
	
  myList[mid],myList[end] = myList[end],myList[mid]
  return mid
\end{lstlisting}

\subsection{Analisi Quick Sort}
Kompleksitas dari \textit{quick sort} tergantung pada  seimbang atau tidak seimbang pembagian partisi pada list yang akan diurutkan. Sebuah partisi yang  baik membagi list menjadi dua list dengan ukuran sama, sedangkan partisi yang buruk membagi list menjadi dua array dengan ukuran yang sangat berbeda. Partisi terburuk menempatkan hanya satu elemen dalam satu list dan semua elemen lainnya dalam list lain. 

Untuk menganalisi kompleksitas dari \textit{quick sort}, terlebih dahulu, perlu diperhitungkan kompleksitas dari algoritma partisi yang digunakan karena algoritma partisi yang digunakan bukan pernyataan sederhana.

\lstset{language=Python}
\label{lst:Partition}
\begin{lstlisting}[frame=single]
def partition(myList,start,end):
 pIdx = random.randrange(start,end) 
 #cost = a
 pivot = myList[pIdx]
 #cost = b
 
 myList[pIdx],myList[end] = myList[end],myList[pIdx]
 #cost = c 
 mid = start
 #cost = d

 for i in range(start,end-1):
  if(myList[i] < pivot):
	 myList[i],myList[mid] = myList[mid],myList[i]
	 mid = mid+ 1
 #cost = i.n+e

 myList[mid],myList[end] = myList[end],myList[mid]
 #cost = f
 return mid
 #cost = g
\end{lstlisting}

Didalam algoritma partisi yang digunakan, terdapat sebuah loop for yang diikuti oleh statement sederhana lainnya. Statement sederhana tersebut dapat diberikan $f(n)= a+b+c+d+f+g$, kecuali yang berada di dalam statement for, dimana nilai $f(n) = i.n+e /O(n)$. Sehingga didapatkan kompleksitas untuk algoritma partisi diatas adalah :

$$ \mathnormal{f(n) = a+b+c+d+i.n+e+f+g} $$
$$ \mathnormal{f(n) = i.n} $$

Setelah mendapatkan nilai kompleksitas dari algoritma partisiya, maka dapat dihitung kompleksitas untuk algoritma \textit{quick sort} .

\lstset{language=Python}
\label{lst:QuickSort}
\begin{lstlisting}[frame=single]
def quickSort(myList, start, end):
 if start < end: #cost = a
   pivot = partition(myList, start, end) #cost= i.n
   quicksort(myList, start, pivot-1) #cost = f(n/2) + b
   quicksort(myList, pivot+1, end) #cost = f(n/2) + c
   return myList #cost = d
\end{lstlisting}

yang secara matematis dapat dituliskan seperti berikut:
$$	  \mathnormal{f(n) = a+i.n+f(\frac{n}{2})+b+f(\frac{n}{2})+c+d } $$
$$	  \mathnormal{f(n) = 2f(\frac{n}{2})+i.n } $$

dengan syarat $n > 1$, maka:
$$	  \mathnormal{f(n) = 2f(\frac{n}{2})+i.n } $$
$$	  \mathnormal{f(n) = 2(2f(\frac{n}{4}))+i.\frac{n}{2} } + i+n $$
$$	  \mathnormal{f(n) = 4f(\frac{n}{4})+2i.n } $$
$$	  \mathnormal{f(n) = 2(4f(\frac{n}{8}))+i.\frac{n}{4} }+2i.n $$
$$	  \mathnormal{f(n) = 8f(\frac{n}{8})+3i.n } $$
$$	  \mathnormal{f(n) = 16f(\frac{n}{16})+4i.n } $$
$$	  \mathnormal{f(n) = 2^{k}f(\frac{n}{2^{k}})+k.i.n } $$

untuk $ \mathnormal{f(n) = (\frac{n}{2^{k}})} $ :

$$ \mathnormal{\frac{n}{2^{k}}} $$
$$ \mathnormal{2^{k}=n} $$
$$ \mathnormal{k=log_{2}n} $$

sehingga :

$$	  \mathnormal{f(n) = 2^{log_{2}n}f(1)+log_{2}n.i.n } $$

$$	  \mathnormal{f(n) = i.n\ log\ n = O(n\ log\ n)} $$

untuk kasus terburuk, jika partisinya dibagi tidak sama rata atau hanya satu elemen dalam satu list dan semua elemen lainnya dalam list lain seperti Figur \ref{fig:WorseCaseQuickSort}, maka kompleksitasnya adalah:

$$	  \mathnormal{f(n) = f(n-1)+i+n} $$
$$	  \mathnormal{f(n) = f(n-2)+i(n-1)+(i.n)} $$
$$	  \mathnormal{f(n) = f(n-2)+2(i.n)} $$
$$	  \mathnormal{f(n) = f(n-3)+i(n-2))+2(i.n)} $$
$$	  \mathnormal{f(n) = f(n-3)+3(i.n)} $$
$$	  \mathnormal{f(n) = f(n-4)+4(i.n)} $$
$$	  \mathnormal{f(n) = f(n-k)+k(i.n)} $$

untuk $ \mathnormal{f(n) = k.n  }$:
$$ \mathnormal{ k.n = 1  }$$
$$ \mathnormal{ k = n }$$

sehingga :

$$ \mathnormal{f(n) = f(1) + n.i.n} $$
$$ \mathnormal{f(n) = i.n^{2} = O(n^{2})} $$
		

\begin{figure}[htbp]
\begin{center}
	\includegraphics[scale=0.5]{QuickSort3.jpg}%
	\caption{Quick Sort Worse Case}%
	\label{fig:WorseCaseQuickSort}%
\end{center}
\end{figure}


Jika partisi seimbang, maka kompleksitas dari  \textit{quick sort} berjalan secepat merge sort $O(n\ log\ n)$. Di sisi lain, jika partisi tidak seimbang, \textit{quick sort} berjalan lambat seperti insertion sort $O(n^{2})$.
\end{document}%ditambahkan

\end{document}
